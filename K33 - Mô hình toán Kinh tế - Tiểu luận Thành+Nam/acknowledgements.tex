\chapter*{LỜI CAM ĐOAN}
\addcontentsline{toc}{chapter}{{\bf LỜI CAM ĐOAN}}
Chúng tôi cam đoan tiểu luận môn học Mô hình Toán trong Kinh Tế, với đề tài Bài toán xếp ba-lô và ứng dụng là báo cáo do Chúng tôi thực hiện dưới sự hướng dẫn của PGS. TS. Nguyễn Lê Hoàng Anh. 
\\\\
Những kết quả nghiên cứu và thực nghiệm của tiểu luận hoàn toàn trung thực và chính xác.
\\\\\\
\begin{flushright}
	
	\begin{tabular}{@{}c@{}}
		\textit{Học viên cao học}\\
		(Ký tên, ghi họ tên)\\\\\\\\ 
        \textbf{Lê Nhựt Nam}
        \\\\\\\\ 
        \textbf{Phạm Thừa Tiểu Thành}
	\end{tabular}
	
\end{flushright}

\thispagestyle{empty}
\chapter*{LỜI CẢM ƠN}
\addcontentsline{toc}{chapter}{{\bf LỜI CẢM ƠN}}

Lời đầu tiên, tôi xin phép gửi lời cảm ơn chân thành đến Thầy hướng dẫn của chúng tôi, PGS. TS. Nguyễn Lê Hoàng Anh - Giảng viên khoa Toán - Tin học, Trưởng bộ môn Tối ưu và Hệ thống, Trường Đại học Khoa học Tự nhiên, Đại học Quốc gia TP. HCM đã trực tiếp giảng dạy và giúp đỡ tận tình trong suốt quá trình học tập môn học và thực hiện tiểu luận. Nhờ vào những định hướng, và góp ý quý giá của thầy, chúng tôi đã hoàn thành trọn vẹn đề tài tiểu luận của mình.
\begin{flushright}
		\begin{tabular}{@{}c@{}}
			Thành phố Hồ Chí Minh, những ngày mùa Thu năm 2024\\\bfseries Lê Nhựt Nam\quad Phạm Thừa Tiểu Thành
		\end{tabular}
\end{flushright}
\thispagestyle{empty}